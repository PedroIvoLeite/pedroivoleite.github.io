%-------------------------
% Resume in Latex
% Author : Pedro Leite
% Based off of: https://github.com/sb2nov/resume
% License : MIT
%------------------------

\documentclass[letterpaper,11pt]{article}

\usepackage{latexsym}
\usepackage[empty]{fullpage}
\usepackage{titlesec}
\usepackage{marvosym}
\usepackage[usenames,dvipsnames]{color}
\usepackage{verbatim}
\usepackage{enumitem}
\usepackage[hidelinks]{hyperref}
\usepackage{fancyhdr}
\usepackage[english]{babel}
\usepackage{tabularx}
\input{glyphtounicode}


%----------FONT OPTIONS----------
% sans-serif
% \usepackage[sfdefault]{FiraSans}
% \usepackage[sfdefault]{roboto}
% \usepackage[sfdefault]{noto-sans}
% \usepackage[default]{sourcesanspro}

% serif
% \usepackage{CormorantGaramond}
% \usepackage{charter}


\pagestyle{fancy}
\fancyhf{} % clear all header and footer fields
\fancyfoot{}
\renewcommand{\headrulewidth}{0pt}
\renewcommand{\footrulewidth}{0pt}

% Adjust margins
\addtolength{\oddsidemargin}{-0.5in}
\addtolength{\evensidemargin}{-0.5in}
\addtolength{\textwidth}{1in}
\addtolength{\topmargin}{-.5in}
\addtolength{\textheight}{1.0in}

\urlstyle{same}

\raggedbottom
\raggedright
\setlength{\tabcolsep}{0in}

% Sections formatting
\titleformat{\section}{
  \vspace{-4pt}\scshape\raggedright\large
}{}{0em}{}[\color{black}\titlerule \vspace{-5pt}]

% Ensure that generate pdf is machine readable/ATS parsable
\pdfgentounicode=1

%-------------------------
% Custom commands
\newcommand{\resumeItem}[1]{
  \item\small{
    {#1 \vspace{-2pt}}
  }
}

\newcommand{\resumeSubheading}[4]{
  \vspace{-2pt}\item
    \begin{tabular*}{0.97\textwidth}[t]{l@{\extracolsep{\fill}}r}
      \textbf{#1} & #2 \\
      \textit{\small#3} & \textit{\small #4} \\
    \end{tabular*}\vspace{-7pt}
}

\newcommand{\resumeSubSubheading}[2]{
    \item
    \begin{tabular*}{0.97\textwidth}{l@{\extracolsep{\fill}}r}
      \textit{\small#1} & \textit{\small #2} \\
    \end{tabular*}\vspace{-7pt}
}

\newcommand{\resumeProjectHeading}[2]{
    \item
    \begin{tabular*}{0.97\textwidth}{l@{\extracolsep{\fill}}r}
      \small#1 & #2 \\
    \end{tabular*}\vspace{-7pt}
}

\newcommand{\resumeSubItem}[1]{\resumeItem{#1}\vspace{-4pt}}

\renewcommand\labelitemii{$\vcenter{\hbox{\tiny$\bullet$}}$}

\newcommand{\resumeSubHeadingListStart}{\begin{itemize}[leftmargin=0.15in, label={}]}
\newcommand{\resumeSubHeadingListEnd}{\end{itemize}}
\newcommand{\resumeItemListStart}{\begin{itemize}}
\newcommand{\resumeItemListEnd}{\end{itemize}\vspace{-5pt}}

%-------------------------------------------
%%%%%%  RESUME STARTS HERE  %%%%%%%%%%%%%%%%%%%%%%%%%%%%


\begin{document}

%----------HEADING----------
% \begin{tabular*}{\textwidth}{l@{\extracolsep{\fill}}r}
%   \textbf{\href{http://sourabhbajaj.com/}{\Large Sourabh Bajaj}} & Email : \href{mailto:sourabh@sourabhbajaj.com}{sourabh@sourabhbajaj.com}\\
%   \href{http://sourabhbajaj.com/}{http://www.sourabhbajaj.com} & Mobile : +1-123-456-7890 \\
% \end{tabular*}

\begin{center}
    \textbf{\Huge \scshape Pedro Ivo Santos Leite} \\ \vspace{1pt}
    \small +55 83-98663-7555 $|$ \href{mailto:x@x.com}{\underline{jake@su.edu}} $|$ 
    \href{https://linkedin.com/in/...}{\underline{linkedin.com/in/jake}} $|$
    \href{https://github.com/...}{\underline{github.com/jake}}
\end{center}

\section*{Professional Summary}
 \begin{itemize}[leftmargin=0.15in, label={}]
    \small{\item{
     Mester em Ciência da Computação (IA e Machine Learning) com sólida trajetória em TI, especializado no desenvolvimento de soluções inteligentes para o Agronegócio. Expertise na criação de modelos preditivos para pecuária de corte e leite, utilizando técnicas avançadas de Machine Learning, Séries Temporais e análise de dados IoT. Perfil híbrido que combina rigor acadêmico com experiência prática em infraestrutura de sistemas, focado em transformar dados complexos em previsões operacionais de alto valor estratégico para o setor agropecuário.
    }}
 \end{itemize}

%-----------EDUCATION-----------
\section{Education}

\resumeSubHeadingListStart
  \resumeSubheading
    {Instituto Federal de Campina Grande}{2025 -- Atual}
    {Master em Inteligência Artificial e Machine Learning}{Campina Grande, PB}
    \resumeItemListStart
        \resumeItem{Pesquisa focada em \textbf{Deep Learning aplicado à Fenotipagem Animal} ou \textbf{Séries Temporais para o Agronegócio}.}
        \resumeItem{Disciplinas: Deep Learning, Estatística Multivariada, Otimização e Processamento de Linguagem Natural.}
    \resumeItemListEnd
%\resumeSubHeadingListEnd

%  \resumeSubHeadingListStart
    \resumeSubheading
      {Universidade Federal da Paraiba}{Campina Grande, PB, Brasil}
      {Bachelor of Arts in Computer Science, Minor in Electronic Engineering}{Jan. 1975 -- Dec 1975}
    \resumeSubheading
      {Universidade Federal da Paraiba}{Campina Grande, PB, Brasil}
      {Post-Graduating in Computer Science }{Jan 1980 -- Dec 1980}
  \resumeSubHeadingListEnd

%-----------EXPERIENCE-----------
\section{Experience}
  \resumeSubHeadingListStart

    \resumeSubheading
      {ITFPB - Machine Learning Researcher (Master's)}{2024 -- Present}
      {Precision Livestock Project}{Campina Grande, Brazil}
      \resumeItemListStart
        \resumeItem{Developing \textbf{Time Series Forecasting} models for milk production and cattle weight gain prediction.}
        \resumeItem{Processing high-volume data from IoT sensors and historical farm records for agribusiness optimization.}
        \resumeItem{Applying \textbf{Feature Engineering} techniques to biological and climatic variables, including the Temperature Humidity Index (THI).}
      \resumeItemListEnd

    \resumeSubheading
      {Intelicampo - Inteligent Farming}{2013 -- Present}
      {IT Project Manager}{Brazil}
      \resumeItemListStart
        \resumeItem{Leading IT initiatives in the \textbf{Agribusiness} sector, focusing on digital transformation and data-driven solutions for farm management.}
        \resumeItem{Overseeing project lifecycles to integrate technology into agricultural operations, enhancing productivity and operational control.}
      \resumeItemListEnd

    % \resumeSubheading
    %   {Upplify Inc}{2013 -- 2023}
    %   {IT Project Manager}{Brazil}
    %   \resumeItemListStart
    %     \resumeItem{Managed a diverse portfolio of IT projects within the \textbf{Agribusiness} industry for a decade, ensuring alignment between technology and business goals.}
    %     \resumeItem{Directed cross-functional teams to deliver scalable software solutions and infrastructure improvements.}
    %   \resumeItemListEnd

    \resumeSubheading
      {RBC Investor Services - Worldwide Banking}{2007 -- 2013}
      {Systems Analyst / IT Project Manager}{Canada}
      \resumeItemListStart
        \resumeItem{Acted as both Systems Analyst and Project Manager in the \textbf{Financial Services} industry, managing complex systems for international investors.}
        \resumeItem{Optimized financial software workflows and ensured high availability of mission-critical systems in a global banking environment.}
      \resumeItemListEnd

    \resumeSubheading
      {Companhia Hidro Elétrica do São Francisco) - Energy Management System}{1978 -- 1999}
      {IT Systems Developer}{Brazil}
      \resumeItemListStart
        \resumeItem{Developed core IT systems for the \textbf{Energy} sector, contributing to the automation and digitalization of large-scale infrastructure.}
        \resumeItem{Built and maintained software solutions for utility management and operational data processing.}
      \resumeItemListEnd

  \resumeSubHeadingListEnd

%-----------PROJECTS-----------
\section{Projects}
    \resumeSubHeadingListStart
      \resumeProjectHeading
          {\textbf{AgroPredict: ML para Desempenho Zootécnico} $|$ \emph{Python, XGBoost, Scikit-learn, Streamlit}}{2024}
          \resumeItemListStart
            \resumeItem{Desenvolvimento de modelos de \textbf{Machine Learning} para prever o Ganho de Peso Médio Diário (GMD) em bovinos de corte e a produção diária em sistemas leiteiros.}
            \resumeItem{Implementei técnicas de \textbf{Feature Engineering} para integrar dados de sensores térmicos (THI) e nutricionais, aumentando a precisão do modelo em 18\% em comparação a métodos estatísticos tradicionais.}
            \resumeItem{Utilizei \textbf{XGBoost Regressor} com otimização de hiperparâmetros (Bayesian Optimization) para lidar com a sazonalidade e ruídos dos dados de campo.}
            \resumeItem{Construí um dashboard interativo em \textbf{Streamlit} para visualização de KPIs operacionais e simulações de cenários de produtividade para gestores rurais.}
          \resumeItemListEnd

      \resumeProjectHeading
          {\textbf{Pipeline de Dados Meteorológicos para IoT} $|$ \emph{Python, FastAPI, PostgreSQL, Docker}}{2024}
          \resumeItemListStart
            \resumeItem{Estruturei um sistema de captura automatizada de dados via APIs meteorológicas para enriquecer bases de dados de fazendas parceiras.}
            \resumeItem{Apliquei conhecimentos de \textbf{Engenharia de Software (TI)} para garantir a escalabilidade do sistema e a integridade dos dados coletados em tempo real.}
          \resumeItemListEnd
    \resumeSubHeadingListEnd

%
%-----------PROGRAMMING SKILLS-----------
\section{Technical Skills}
 \begin{itemize}[leftmargin=0.15in, label={}]
    \small{\item{
     \textbf{Languages}{: Java, Python (Pandas, Scikit-learn), C/C++, SQL (Postgres), JavaScript, HTML/CSS, R} \\
     \textbf{Frameworks}{: React, Node.js} \\
     \textbf{TI/MLOps:} Git} \\
     \textbf{Libraries}{: pandas, NumPy, Matplotlib} \\
     \textbf{Agro Context:} Ganho de Peso Médio Diário (GMD), Curvas de Lactação, Análise de Estresse Térmico.}
    }}
 \end{itemize}


%-------------------------------------------
\end{document}
